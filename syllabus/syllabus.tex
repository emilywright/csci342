\documentclass{article}
\usepackage[margin=0.75in]{geometry}
\usepackage{alltt}
\usepackage{multicol}
\usepackage{hyperref}
\begin{document}

\centerline{\large \bf Syllabus}

\centerline{\bf CSCI 342, Web Scripting, Fall 2016}

\begin{itemize}

\item
{\bf Instructor:} Geoffrey Matthews, x3797, {\tt  geoffrey dot matthews at wwu dot edu}
\item
{\bf Office hours:} MTWF 10:00, CF 469
\item
{\bf Text:} \begin{itemize}\item
Web Programming Step by Step: \url{http://www.webstepbook.com/}
\item
  Other online readings as assigned:
  \begin{itemize}
  \item \url{http://www.w3schools.com/jquery/}
  \item \url{https://docs.djangoproject.com/en/1.10/intro/tutorial01/}
  \item \url{https://www.tutorialspoint.com/nodejs/}
  \end{itemize}
\end{itemize}
\item {\bf Webpage:}  \url{wwu.instructure.com}:  homework assignments, grades
\item {\bf Repository:} \url{https://github.com/geofmatthews/csci342}:  handouts, lectures, code
\item
{\bf Lectures:} 
CF023, MTWF 11:00

\item
{\bf Content:} We will study the development of programs and scripts
for Web server applications and implement dynamic web pages in the
Unix and Windows environments.  The course will include a study of
various scripting languages, and database access through database
management systems.  Implementations will be using a variety of
languages (possibly including PHP, Python, Ruby, Javascript and
others), their database interfaces and a simple SQL database
management system.

\item {\bf Course goals:}  After successfully completing this course,
  students will be able to:
  \begin{enumerate}
    \item have an understanding of what scripting languages can do and
      how they are used to develop Web applications
    \item
      demonstrate a
      working knowledge of a variety of scripting languages.
    \item
      have a complete understanding of
      Web concepts and technologies such as HTTP protocol, CGI,
      database connectivity, security AJAX and XML
    \item have a working
      knowledge of the design and development of web based
      applications using a number of tools and strategies
    \item have the
      ability to use of databases as data repositories for web
      applications
    \item implement a number of tools and technologies
      used today to develop modern web applications
  \end{enumerate}
  

\item {\bf Software:}
  The software we develop will require both scripts that run in
  web browsers and scripts that run in web
  servers.  You already have a browser, but it will be convenient to
  have your own server, as well.

  You may already have this on your system, but if not
you should install on your laptop (or on a portable disk or thumb drive)
a copy of XAMPP: \url{https://www.apachefriends.org/index.html}.  This
will install an Apache web server complete with PHP, Perl, and MariaDB
(similar to MySQL).  We will discuss how to get other
languages/environments running later.

\item {\bf Exams:}   One
  midterm and one final.  You may bring two double-sided pages of notes to use
  during the exams.


\item {\bf Homework:}  Homework assignments will be passed out
  regularly through the quarter, generally involving the creation of a
  web site.   Homework will be due at midnight
  on the due date.  Late work is accepted at a penalty
  of 25\% per each fraction of 24 hours late.  There may be an assignment
  due during dead week.


\item {\bf Grading:} \\
$0\leq F < 60 \leq D < 70 \leq C < 80 \leq B < 90 \leq A$\hfill
\begin{tabular}{|l|l|l|}\hline
Homework  & Midterm & Final\\\hline
50\% & 20\% & 30\%\\\hline
\end{tabular}

\item {\bf Academic dishonesty:} Academic dishonesty policy and
  procedure is discussed in the University Catalog, Appendix D.  All
  students should read this section of the catalog.  Academic
  dishonesty consists of misrepresentation by deception or other
  fraudulent means.  In computer science courses this frequently takes
  the form of copying another's program, either a fellow student's
  program, or copying one from the web.  Due diligence should be
  exercised in the labs at all times, since both copying and letting
  someone else copy your program are equally culpable.  Do not walk
  away from your computer in the lab without logging out or locking
  the screen.  Do not share files, even if it is just to ``show them
  something.''  Describe it in words, or talk to them in person, never
  share code.

\item {\bf Collaboration:} Collaboration with your fellow students is
  a good way to learn.  Feel free to share ideas, solve problems, and
  discuss your programs with other students.  However, collaboration
  is {\em not} copying.  All code should be original.  Remember the
  {\bf Simpson's Rule:} after discussing homework with another
  student, each of you must destroy all written notes, pictures,
  files, {\em etc.} that you shared.  After that, you must watch a
  rerun of {\em the Simpson's}, or do something else unrelated, for
  half an hour.  Then you can take the knowledge you gained from
  another student and put it to work, since it is now not copying, but
  learning.  You have made it your own.

\item
{\bf Approximate Schedule:} The following schedule may be adjusted
radically depending on interests and problems as they occur.  


\begin{verbatim}
   September 2016     
Su Mo Tu We Th Fr Sa  
18 19 20 21 22 23 24  Chapters 1-4
    October 2016      
25 26 27 28 29 30  1  Chapters 5-6
 2  3  4  5  6  7  8  Chapters 7-8
 9 10 11 12 13 14 15  Chapters 9-10
16 17 18 19 20 21 22  Chapters 11-12
23 24 25 26 27 28 29  Review and Catchup, Midterm Friday October 28
   November 2016      
30 31  1  2  3  4  5  jQuery
 6  7  8  9 10 11 12  Chapters 13-14
13 14 15 16 17 18 19  Chapters 15-16
20 21 22 23 24 25 26  NodeJS,  Django
   December 2016      
27 28 29 30  1  2  3  Review and Catchup
 4  5  6  7  8  9 10  Final Wednesday December 7 8:00am
\end{verbatim}
\end{itemize}

\end{document}
