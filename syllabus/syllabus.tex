\documentclass{article}
\usepackage[margin=0.75in]{geometry}
\usepackage{alltt}
\usepackage{multicol}
\usepackage{hyperref}
\begin{document}

\centerline{\large \bf Syllabus}

\centerline{\bf CSCI 342, Web Scripting, Fall 2017}

\begin{description}

\item[Instructor:] Geoffrey Matthews, x3797, {\tt geoffrey dot
  matthews at wwu dot edu}
\item[Lectures:] MTWF 10:00-10:50 am AH 018
\item[Office hours:] MTWF 11:00, CF 469
\item[Webpage:] \url{wwu.instructure.com}: homework assignments,
  grades
\item[Repository:] \url{https://github.com/geofmatthews/csci342}:
  handouts, lectures, code
\item[Text:] There is no textbook. Online readings are listed in the
  {\tt lessonplan.pdf} file on github.

\item[Catalog copy:] Development of dynamic Web applications. Study of
  various server-side scripting languages (PHP, Perl, Python, Ruby on
  Rails) for creating dynamic Web pages and querying and manipulating
  Databases on the Web.

\item[Content:] We will study the development of programs and scripts
for Web server applications and implement dynamic web pages in the
Unix and Windows environments.  The course will include a study of
various scripting languages, and database access through database
management systems.  Implementations will be using a variety of
languages (possibly including PHP, Python, Ruby, Javascript and
others), their database interfaces and a simple SQL database
management system.

\item[Course goals:] After successfully completing this course,
  students will be able to:
  \begin{enumerate}
  \item have an understanding of what scripting languages can do and
    how they are used to develop Web applications
  \item
    demonstrate a working knowledge of a variety of scripting
    languages.
  \item
    have a complete understanding of Web concepts and technologies
    such as HTTP protocol, CGI, database connectivity, security, AJAX
    and XML
  \item have a working knowledge of the design and development of web
    based applications using a number of tools and strategies
  \item have the ability to use of databases as data repositories for
    web applications
  \item implement a number of tools and technologies used today to
    develop modern web applications
  \end{enumerate}
  
  
\item[Software:] The software we develop will require both
  scripts that run in web browsers and scripts that run in web
  servers.  You already have a browser, but it will be convenient to
  have your own server, as well.

  You may already have this on your system, but if not you should
  install on your laptop (or on a portable disk or thumb drive) a copy
  of XAMPP: \url{https://www.apachefriends.org/index.html}.  This will
  install an Apache web server complete with PHP, Perl, and MariaDB
  (similar to MySQL).

  NodeJS and Django each come with their own demo servers, available
  for download on their sites.

\item[Exams:]   You may bring two
  double-sided pages of notes to use during the exams.

  \begin{tabular}{|l|l|}\hline
    Midterm &  Friday, November 3, classtime  \\\hline
Final   & Tuesday, December 12, 8:00-10:00am \\\hline
    \end{tabular}

\item[Pop quizzes:]  Pop quizzes will be handed out from time to
  time in class.  You will work on them individually or in small
  groups, and then, before handing them in, we will solve them in
  class.  They will be turned in at the end of class.

\item[Homework:] Homework assignments will be passed out regularly
  through the quarter, generally involving the creation of a web site.
  Homework will be due at midnight on the due date.  Work will be
  accepted with no penalty up to 48 hours after the due date.  These
  48 hours constitute an extension to the due date for emergency
  purposes only.  Do not abuse them: there will be no further
  extensions.  There may be an assignment due during dead week.

\item[Grading:]  Adjusting the grades upward based on class
  statistics, and awarding plus and minus grades, are at the
  discretion of the instructor.

\begin{tabular}{|c|c|c|c|}\hline
Pop Quizzes & Homework  & Midterm & Final\\\hline
10\% & 40\% & 20\% & 30\%\\\hline
\end{tabular}\hfill
  \begin{tabular}{|r|c|c|c|c|c|}\hline
    Percentage: & 0--59 & 60--69 & 70--79 & 80--89 & 90--100\\\hline
    Grade: & F & D & C & B & A \\\hline
    \end{tabular}

\item[Academic dishonesty:] Academic dishonesty policy and
  procedure is discussed in the University Catalog, Appendix D.  All
  students should read this section of the catalog.  Academic
  dishonesty consists of misrepresentation by deception or other
  fraudulent means.  In computer science courses this frequently takes
  the form of copying another's program, either a fellow student's
  program, or copying one from the web.  Due diligence should be
  exercised in the labs at all times, since both copying and letting
  someone else copy your program are equally culpable.  Do not walk
  away from your computer in the lab without logging out or locking
  the screen.  Do not share files, even if it is just to ``show them
  something.''  Describe it in words, or talk to them in person, never
  share code.

\item[Collaboration:] Collaboration with your fellow students is
  a good way to learn.  Feel free to share ideas, solve problems, and
  discuss your programs with other students.  However, collaboration
  is {\em not} copying.  All code should be original.  Remember the
  {\bf Simpson's Rule:} after discussing homework with another
  student, each of you must destroy all written notes, pictures,
  files, {\em etc.} that you shared.  After that, you must watch a
  rerun of {\em the Simpson's}, or do something else unrelated, for
  half an hour.  Then you can take the knowledge you gained from
  another student and put it to work, since it is now not copying, but
  learning.  You have made it your own.




\end{description}

\end{document}
